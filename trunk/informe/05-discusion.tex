\section{Discusi\'on}

\subsection{Colores contra cuadrados de colores}
Cuando probamos con la primer imagen de colores, vecinos resultó ser el que mejor resultados tuvo en todos los sentidos. Esto nos llamó mucho la atención ya que es un algoritmo muy simple para superar a otros desarrollados en distintos papers.\\
Por eso decidimos hacer una segunda prueba con una imagen que tenga la misma idea pero más compleja (con límites tambien horizontales, curvos y con la marca de agua) y ahí si obtuvimos resultados mas acordes.\\
En nuestra implementación vecinos para calcular el color de un pixel se fija en el inmediatemetne superior e inferior. Como en la primera imagen los límites existían únicamente de forma vertical no influían en esta lógica. Por eso fué que funcionó tan bien en el primer caso. En cambio en el segundo fué mucho más ineficiente por las complejidades antes mencionadas.
